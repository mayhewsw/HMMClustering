% Template for ICME-2010 paper; to be used with:
%          spconf.sty  - ICASSP/ICIP LaTeX style file, and
%          IEEEbib.bst - IEEE bibliography style file.
% --------------------------------------------------------------------------
\documentclass{article}
\usepackage{spconf_ICME,amsmath,epsfig,fancyhdr}
\setlength{\paperwidth}{215.9mm} \setlength{\hoffset}{-9.7mm}
\setlength{\oddsidemargin}{0mm} \setlength{\textwidth}{184.3mm}
\setlength{\columnsep}{6.3mm} \setlength{\marginparsep}{0mm}
\setlength{\marginparwidth}{0mm} \setlength{\paperheight}{279.4mm}
\setlength{\voffset}{-7.4mm} \setlength{\topmargin}{0mm}
\setlength{\headheight}{0mm} \setlength{\headsep}{0mm}
\setlength{\topskip}{0mm} \setlength{\textheight}{235.2mm}
\setlength{\footskip}{12.4mm} \setlength{\parindent}{1pc}

\usepackage{minted}

\ICMEfinalcopy % *** Uncomment this line for the final submission

\def\ICMEPaperID{****} % *** Enter the ICME Paper ID here
\def\httilde{\mbox{\tt\raisebox{-.5ex}{\symbol{126}}}}

% Pages are numbered in submission mode, and unnumbered in camera-ready
\ifICMEfinal\pagestyle{empty}\fi


\begin{document}\sloppy

% Example definitions.
% --------------------
\def\x{{\mathbf x}}
\def\L{{\cal L}}


% Title.
% ------
\title{Using N-gram similarity metrics for Relation Clustering}
%
% Single address.
% ---------------
\name{Stephen Mayhew, Nicholas Kamper}

\address{
\textit{mayhewsw@rose-hulman.edu}\\
\textit{kampernj@rose-hulman.edu}\\
Rose-Hulman Institute of Technology}


\maketitle
% insert page header and footer here for IEEE PDF Compliant
\thispagestyle{fancy} \fancyhead{} \lhead{}
\lfoot{\copyright2012 SWM, NJK} \cfoot{}
%\rfoot{ICME 2010}
\renewcommand{\headrulewidth}{0pt}
\renewcommand{\footrulewidth}{0pt}


%
%\begin{abstract}
%This paper has to do with relation extraction. This is probably best written %after the rest of the paper is done.
%\end{abstract}
%
%\begin{keywords}
%Natural Language Processing, Information Extraction, Relation Extraction, %Coreference Resolution
%\end{keywords}
%
\section{Introduction}
\label{sec:intro}

Our idea was to use Hidden Markov Models (HMM) to cluster relations.
Relations come from ReVerb from University of Washington. 
Problem can be summarized with:

Given relations, of the form (object, relation, object), understand which extractions have the same meaning. For example, the following extractions would all be clustered together

\fvset{frame=single}
\begin{minted}[fontsize=\footnotesize]{text}
(tony blair, is prime minister of, britain) 
(tony blair, is prime minister of, great britain) 
(tony blair, is the prime minister of, the UK)
\end{minted}
\fvset{frame=none}

We note that facts do not need to be true to be part of the system.



\section{Experiments}

Use HMM for the actual clustering. Dead wall.

HMM for scoring (Multiple sequence alignment). Dead wall.

Decided to ditch HMM altogether.

How about using N-grams? (Considered using Google data)
Started writing our own code in Java (because of legacy code)

Going slowly, tried several libraries: NLTK, OpenNLP, KYLM, and some other small open source projects.

Finally, we chanced on Python ngram similarity library that uses ngrams to predict similarity between two strings. 

``The NGram class is a set that supports searching for its members by N-Gram string similarity." \cite{py_ngram_lib}


\begin{thebibliography}{1}

\bibitem{py_ngram_lib}
Michel Albert, Graham Poulter, and open source contributors. Python NGram Similarity Library. \textit{http://packages.python.org/ngram/}
\end{thebibliography}


\end{document}